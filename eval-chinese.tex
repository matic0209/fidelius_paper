\section{评估}

在本节中,我们通过综合实验评估Fidelius的性能。首先,我们通过分别执行CPU密集型和I/O密集型任务来评估其时间消耗。随后,我们对Fidelius与在以太坊虚拟机(EVM)中执行分析程序的替代解决方案进行性能比较。我们的实验在一台配备Intel(R) Core(TM) i5-10400F CPU的机器上进行,该CPU拥有12个核心,频率为2.9 GHz,16 GB内存和12 MB缓存。除非另有说明,每个实验重复1,000次以计算平均值。

\subsection{CPU密集型任务}
为了评估Fidelius在CPU密集型任务上的性能,我们实现了一个具有三层和128个隐藏单元的神经网络算法,设计用于使用MNIST数据集识别手写数字。我们报告算法的时间消耗和准确性。此外,我们在原始CPU(无SGX)上部署算法进行比较。基于Fidelius的算法与基于原始CPU的算法之间的主要差异包括:
\begin{itemize}
  \item 随机库。基于Fidelius的算法使用基于Intel SGX的随机库,而基于原始CPU的算法依赖C++随机库。
  \item 数据解密。Fidelius在启动算法前需要解密密封数据,产生额外的处理时间。相反,基于原始CPU的算法可以直接访问明文数据。
\end{itemize}
对于Fidelius和原始CPU,我们保持固定的测试集数量为1,000,同时将训练集数量从10,000逐步增加到50,000。对于每个训练集,我们分别将epoch设置为100和200来执行算法。

图说明了在不同epoch下在Fidelius和原始CPU上运行的算法的时间消耗。随着训练集数量的增加,Fidelius和原始CPU上算法的时间消耗都呈现线性增长趋势。然而,这两种算法之间的时间消耗差异很小。具体而言,观察到Fidelius上的执行时间仅比原始CPU长2\%,主要是由于数据解密开销。

图显示了在不同epoch下Fidelius和原始CPU上算法的准确性,两者都达到了超过91\%的准确性。在100个epoch时,准确性略有波动,但平台之间没有明显差异。在200个epoch时,两种算法都收敛到几乎相同的准确性。Fidelius与原始CPU相比时间消耗增加了2\%,但保持了相似的准确性水平,在CPU密集型任务中表现出可比的性能。

\subsection{I/O密集型任务}
为了评估Fidelius在处理I/O密集型任务时的时间效率,我们在Fidelius和原始CPU平台上测试了一个在超过1 GB文件中搜索目标字符串的算法。时间消耗取决于数据加载和字符串搜索,Fidelius由于数据解密需要额外时间。考虑到Intel SGX在Fidelius中的约束,我们将数据加载块大小设置为64 KB和256 KB,并为每个块大小改变数据行数从1到128和1到256来测量时间效率。

图显示了Fidelius与原始CPU在不同数据块大小下的时间效率比较。随着读取行数的增加,Fidelius的时间消耗显著减少。例如,在64 KB块大小时,Fidelius的时间从81秒(1行)下降到11秒(128行)。类似地,在256 KB块大小时,从213秒(1行)减少到10秒(256行)。将块大小增加到256 KB进一步优化了Fidelius的性能。尽管有所改进,Fidelius与原始CPU之间的时间消耗仍存在差距,主要是由于I/O密集型任务中的解密过程。

\subsection{性能比较}
在本节中,我们将Fidelius与SDTE的性能进行比较,SDTE使用\textit{k}-最近邻(\textit{k}-NN)算法作为以太坊智能合约在以太坊虚拟机(EVM)上执行机器学习任务。我们在Fidelius和原始CPU上实现\textit{k}-NN算法来评估每个解决方案的时间消耗,使用来自Kaggle的Titanic数据集作为输入。

为了评估\textit{k}-NN算法在以太坊虚拟机(EVM)上的执行时间,我们在Ganache(个人以太坊区块链)上部署了\textit{k}-NN智能合约。报告的EVM上\textit{k}-NN的时间消耗仅关注执行时间,不包括交易广播或提交所花费的时间。我们还排除了将Titanic数据集上传到区块链所需的时间,因为\textit{k}-NN合约使用链上数据。由于EVM的内存限制,\textit{k}-NN算法无法处理超过60个测试集,因此我们将测试集限制在10-60个,同时保持训练集为100个。

图显示了\textit{k}-NN算法在原始CPU、Fidelius和以太坊虚拟机(EVM)上的时间消耗,后者比其他的大约长30倍。此外,由于内存约束,EVM执行在测试超过60个集时经常失败,突出了在EVM上运行具有大数据集的复杂机器学习算法的困难,正如之前关于以太坊内存限制的研究所指出的那样。相比之下,Fidelius为复杂的机器学习任务提供了更可靠和稳定的环境。 