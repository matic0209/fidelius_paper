\section{Threat Model and Design Choice}

In this section, we first introduce the roles within the system, then detail the threat model, highlighting potential risks and discussing various types of attacks. Additionally, we establish key assumptions that form the foundation for the design and implementation of Fidelius. Subsequently, we delve into the design choices made for Fidelius, aimed at mitigating the identified risks and countering potential attacks on the system.

\subsection{Roles}
\begin{itemize}
    \item Data Provider (DP). Data provider, as the sole owner of the raw data, initially publishes metadata on the blockchain, which includes the hash and a necessary description of the raw data. The accuracy of this metadata is validated by the data provider's credibility, exemplified by a history of successful data analyses.
    \item Model Provider (MP). Model provider supplies analysis programs for specific types of data.
    \item Cloud Service Provider (CSP). Cloud computing provider supplies the computational resources required for data analysis and additionally provides a Trusted Execution Environment to ensure secure data analysis. After receiving a request for a data analysis task, the cloud computing provider is obligated to execute the analysis program and return the results to the blockchain.
    \item Data User (DU). Data user selects the desired raw data by examining the metadata on the blockchain and initiates a data analysis to obtain the results from the specified analysis program.
    \item Blockchain. Blockchain serves as a reliable, failure-resistant third party for data transmission and storage. Specifically, a smart contract verifies the correctness of the signature on the analysis results.
\end{itemize}

\subsection{Threat Model}\label{subsec:threatmodel}
% 系统包含数据提供方、模型提供方、云服务提供方以及数据适用方四个角色,四个角色之间是相互不信任的。

The system comprises four distinct roles: data provider, model provider, cloud service provider, and data user, which are inherently distrustful of one another.
% 在数据分析的过程中,存在以下攻击:
% 数据窃取攻击:模型提供方的模型逻辑存在窃取原始数据或者中间数据的恶意代码;云服务提供方通过提供的硬件或软件资源窃取数据;数据使用方登录进服务器窃取数据。
% 数据伪造攻击:数据提供方提供的数据和声称的不一致。
% 数据滥用攻击:数据提供方的数据被使用在不同的模型、不同的云服务器上。
% 结果伪造攻击:数据分析的结果并不是模型如实运行得到的。
% 结果窃取攻击:数据分析的结果被数据提供方、模型提供方、云服务提供方或未知的攻击者获取。
During the data analysis process, the following attacks are prevalent:
\begin{itemize}
    \item Data Theft Attacks: Cloud service provider may steal data through the hardware or software resources they provide; data user may log into the server to steal data. The program provided by the model provider may contain malicious code aimed at stealing raw or intermediate data.
    \item Data Fabrication Attacks: The data provided by the data provider is inconsistent with what is claimed.
    \item Data Misuse Attacks: Data from the data provider is used across different models and cloud servers without proper authorization.
    \item Result Fabrication Attacks: The results of the data analysis do not accurately reflect the true execution of the model.
    \item Result Theft Attacks: The results of data analysis are illicitly accessed by the data provider, model provider, cloud service provider, or an unidentified attacker.
\end{itemize}

We do not address side-channel attacks on Intel SGX in this paper. We assume Intel SGX's hardware functionality is as advertised, ensuring that code within an enclave remains unaltered and the values of internal variables are protected from direct memory access. Notably, our reliance on Intel is limited to a one-time interaction during the setup process; no further contact with Intel servers or DCAP is necessary. In contrast, previous approaches rely on remote attestation, requiring ongoing integrity and availability of Intel servers or DCAP for each data analysis task.

We assume the presence of a collision-resistant hash function and secure signature and encryption schemes. Specifically, our signatures incorporate a \textit{nonce} to ensure that a valid signature pair $(msg, signed_{msg})$ cannot be generated by an adversary without the private key, rendering historical signatures invalid.


\subsection{Design Choice}\label{subsec:designchoice}
To counter the identified attacks, we have made the following design choices.
% 抵御数据窃取攻击:所有存在于非可信环境(例如网络传输、云服务器的非TEE部分)的数据都经过加密处理,以防止数据使用方或云服务提供商直接窃取数据。模型提供方提供的分析程序,在进行数据分析前进行代码静态分析,防止模型提供方通过数据分析结果窃取数据。
% 抵御数据伪造攻击:执行数据分析前,检查数据哈希与声称的哈希是否一致。
% 抵御数据滥用攻击:引入数字签名技术,数据提供方用私钥签名使用数据的平台、模型以及模型的参数,在执行数据分析前,会验证该签名的有效性。
% 抵御结果伪造攻击:将数据使用方的私钥加密地传输到云服务器的可信执行环境中,并用这个私钥签名结果。当数据使用方拿到结果时,也会得到该私钥对结果的签名,若签名验证通过,说明结果确实是由可信执行环境中得到的,故结果未经伪造。这一思想类似于零知识证明。
% 抵御结果窃取攻击:数据分析结果由数据使用方的公钥加密,只能被数据使用方查看。
\begin{itemize}
    \item Defending Against Data Theft Attacks: All data located in non-trusted environments, such as network transmissions or the non-TEE components of cloud servers, are encrypted to prevent direct data theft by data users or cloud service providers. Additionally, the analysis programs provided by model providers undergo static code analysis prior to data analysis to prevent data theft through the analysis results.
    \item Defending Against Data Fabrication Attacks: Prior to conducting data analysis, the data hashes are verified to ensure they match the claimed hashes.
    \item Defending Against Data Misuse Attacks: Digital signature technology is implemented, enabling the data provider to sign the platform, model, and model parameters using a private key. The validity of this signature is verified before executing data analysis.
    \item Defending Against Result Fabrication Attacks: The data user's private key is securely transmitted to the Trusted Execution Environment (TEE) on the cloud server and used to sign the results. When the data user receives the results, they also receive the signature made by the private key. If the signature is validated, it confirms that the results were indeed obtained from the TEE, ensuring the results have not been fabricated. This approach is analogous to zero-knowledge proof.
    \item Defending Against Result Theft Attacks: The results of data analysis are encrypted with the data user's public key, ensuring that only the data user can access them.
\end{itemize}