\section{引言}
在大数据的时代,人们越来越依赖于数据分析进行决策。例如,当一家超市想要合理进货以便最大化其利润时,他需要考虑超市历史的售卖订单、市场上客户对于商品的喜爱度、各类商品的利润率等方面,这往往需要通过分析各个方面大量的数据来做决定。

为了完成数据分析任务,那些没有服务器的人通常把数据分析的任务托管到云服务器上。但在使用云服务器进行数据分析的同时,人们会担心这些重要的数据遭到泄漏,因为这个数据往往涉及到个人的隐私、商业机密等。

为了避免数据的泄漏的问题,越来越多的相关技术被提出以解决数据泄露的问题。这些技术包括:多方安全计算、同态加密、联邦学习、可信执行环境等。

在这些技术中,MPC和HE具备极高的安全性,因为它是密码学安全的,但由于MPC和HE包含了复杂的密码学操作,其性能相较于其他几类技术表现极差。FL常见于机器学习的场景,主要用于联合建模,对于一般的数据分析场景并不适用。TEE满足各类通用计算的场景,其计算性能较高,但其基于硬件的安全性,相较于MPC和HE的密码学安全,其安全性更低,而且经常出现新发现的基于TEE的侧信道攻击。

然而,随着数据分析的场景越来越复杂,现有的这些技术不能直接应用于那些复杂的场景,来解决数据泄漏的问题。

现有的数据分析场景中,仅存在租户和云计算服务提供商两个角色,现有技术主要是为了防止云计算服务提供商泄漏租户的数据。但是现在的数据分析场景中,包含了数据提供方、数据使用方、模型提供方、云计算服务提供商这些角色,数据需要防止被数据提供方之外的所有角色泄漏。除此之外,还需要维护数据分析场景中各方的权益,例如,数据使用方得到合理的分析结果,数据提供方、模型提供方、云计算服务提供商获得相应的收益。

因此,在针对复杂的数据分析场景时,我们需要检测数据使用方是否通过输出的计算结果中包含敏感信息以泄漏数据,模型提供方是否通过恶意的模型泄漏数据,云计算服务提供商通过篡改hypervisor、操作系统、内存、磁盘等泄漏数据,数据提供方、模型提供方以及云计算服务提供商共同生成的计算结果是否合法。

很多相关的研究被提出,来解决复杂的数据分析场景时所面临的问题,包括SDTE,PrivacyGuard等。SDTE在数据交易行业引入了一种创新的"数据处理即服务"模型。利用这个模型,它通过一系列交易协议实现安全的数据交易。PrivacyGuard利用智能合约定义数据使用策略,并利用TEE进行高效的合约执行,同时保护私有数据。SPDS与PrivacyGuard类似,使用智能合约定义数据使用策略,并实现两阶段交付协议以确保计算结果和支付的安全发布。Amanuensis探索了区块链技术和TEE的交集,以解决数据溯源、机密性和用户隐私在数据共享中的挑战。

然而,这些相关的研究在解决复杂的数据分析场景时,仍存在一些限制。

首先,这些相关的研究工作不能保证数据分析结果中不会泄漏数据。虽然这些工作引入了数据使用规则,用以描述哪些人以什么样的价格访问哪些类型的数据,但对数据进行分析的程序是存在作恶的可能的,分析程序的最终输出结果可能是包含敏感信息泄漏的。一个最简单的例子是,分析程序直接输出原始数据,这样数据分析结果就获取到了完整的原始数据。

一个straightforward的解决方案是,要求分析程序公开,数据提供方可以审核分析程序是否存在泄漏数据的代码。但是对于模型提供方而言,模型也是他们重要的资产,他们不希望模型公开。另一方面,对于复杂的分析程序来说,审核分析程序存在巨大的工作量,不可能要求每一个分析任务都去进行审核,这会大大降低数据分析的效率。

其次,当前的工作中并不能保证计算结果的可信与可验证。具体来说,数据使用方得到计算结果后,他不能确定这个计算结果确实是通过指定的数据与指定的模型运行之后得到的计算结果,而不是一个随机生成的结果。更进一步,数据使用方并没有任何方式去验证这个结果的正确性。如果计算结果的可信和可验证都无法保证,那么完全可以用一个伪造的计算结果来欺骗数据使用方,数据使用方的利益将会受到损害。

最后,当前的工作中每一次进行数据分析任务都需要使用remote attestation来保证分析程序的一致性,但是有相关工作表明,remote attestation具有低效、依赖可信第三方的缺点。以Intel SGX中的Enhanced Privacy ID(EPID)为例,一次remote attestation的过程会涉及到Intel Provisioning Service(IPS)、Intel Attestation(IAS) Service、Intel-signed provisioning enclave(PvE)、Intel-signed quoting enclave(QE)以及分析程序Enclave之间的交互。这些服务和Enclaves之间的交互需要通过广域网传输数据,同时,传输的数据需要加密保证其安全性,对于频繁的数据分析任务而言,性能上会有较大的损失。另一方面,IPS和IAS是Intel提供的中心化服务,remote attestation需要依赖其稳定运行。

在本文中,我们提出了Fidelius,一个利用Intel SGX和区块链来增强数据分析安全性的系统,解决了前面提到的局限性。其中,Intel SGX保证数据分析过程的security和integrity,区块链用于可信的传输、存储和验证。

为了解决计算结果泄漏数据的问题,Fidelius使用了静态二进制分析的方法,检查模型提供方的分析程序是否遵循隐私描述语言。其中,我们引入的隐私描述语言将数据的运算规则描述为有限状态机,反映了从输入数据到输出数据的状态转换,不在隐私描述语言描述的状态转换,都会被认为违反了隐私规则。

为了解决计算结果的可信与可验证的问题,Fidelius提供了一套密码协议,该协议中由数据使用方提供一个私钥,该私钥通过加密转发至分析程序的Enclave中,并签名计算结果。由于该私钥仅在指定的分析程序Enclave中解密获得,数据提供方、模型提供方、云计算服务提供商均无法获取,故只要被签名的计算结果能够通过验证,说明计算结果确实出自指定的分析程序且可验证。

为了解决remote attestation低效、持续依赖中心化服务的问题,Fidelius结合设计的密码协议以及local attestation实现分析程序的一致性验证。Fidelius设计了密钥管理的Enclave,在初始化的过程中获取密钥的授权,在此后的所有数据分析任务中,使用该经授权的密钥执行local attestation完成分析程序的一致性验证。

本文的主要贡献总结如下:
\begin{itemize}
    \item 首先,我们引入了隐私描述语言(PDL)结合静态二进制分析,严格强制执行数据机密性,防止计算结果中的敏感数据泄露。
    \item 其次,我们设计了一种密码协议来确保计算结果的可靠性和可验证性。
    \item 我们集成了密码协议与本地认证机制,确保在受保护环境中执行的分析程序的完整性和正确性。
    \item 最后,我们评估了Fidelius的性能,实验结果表明它产生的开销最小,对数据分析系统的贡献不到2\%,同时性能超越现有解决方案30倍以上。
\end{itemize} 