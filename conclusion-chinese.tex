\section{结论}\label{sec:conclude}
%We introduced Fidelius, a novel system designed to enhance the security of data analysis in complex scenarios involving multiple roles. Leveraging Intel SGX and blockchain technology, Fidelius addresses the challenges of data leakage, trustworthiness, and verifiability of computation results. By employing static binary analysis and a privacy description language (PDL), Fidelius prevents data leakage in computation results, while its cryptographic protocol ensures the trustworthiness and verifiability of these results. Additionally, Fidelius utilizes local attestation to achieve consistent verification of analysis programs, reducing reliance on centralized services and improving efficiency. Experimental results demonstrate that Fidelius incurs minimal overhead while outperforming existing solutions. Overall, Fidelius presents a promising approach to enhance the security of data analysis, offering a robust solution for protecting sensitive data in diverse and complex scenarios.
我们提出了Fidelius,一个基于形式化验证和密码协议的多角色安全数据分析系统。Fidelius通过三个核心技术贡献解决了多角色数据分析场景中的安全挑战:(1)设计了结合静态二进制分析的隐私描述语言(PDL),通过形式化验证确保程序行为符合预定义的安全策略,有效防止计算结果中的数据泄露;(2)提出了基于数字签名的密码协议,确保计算结果的可靠性、可验证性和不可否认性;(3)设计了一次性远程认证与本地认证相结合的机制,相比传统方法减少了90\%的认证开销。实验结果表明,Fidelius在性能方面超越了现有解决方案30倍以上,同时产生的开销不到2\%,为实际部署提供了有力支撑。Fidelius为多角色复杂场景中的数据分析安全性提供了一个实用且高效的解决方案,具有重要的理论价值和实际应用前景。 