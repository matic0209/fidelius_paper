\section{威胁模型与设计选择}

在本节中,我们首先介绍系统中的角色,然后详细说明威胁模型,突出潜在风险并讨论各种类型的攻击。此外,我们建立关键假设,这些假设构成了Fidelius设计和实施的基础。随后,我们深入探讨为Fidelius做出的设计选择,旨在缓解已识别的风险并应对系统上的潜在攻击。

\subsection{角色}
\begin{itemize}
    \item 数据提供方(DP)。数据提供方作为原始数据的唯一所有者,最初在区块链上发布元数据,包括哈希值和原始数据的必要描述。此元数据的准确性通过数据提供方的可信度进行验证,以成功数据分析的历史为例。
    \item 模型提供方(MP)。模型提供方为特定类型的数据提供分析程序。
    \item 云服务提供方(CSP)。云计算提供方提供数据分析所需的计算资源,并额外提供可信执行环境以确保安全的数据分析。在接收到数据分析任务请求后,云计算提供方有义务执行分析程序并将结果返回给区块链。
    \item 数据使用方(DU)。数据使用方通过检查区块链上的元数据选择所需的原始数据,并启动数据分析以从指定的分析程序获得结果。
    \item 区块链。区块链作为数据传输和存储的可靠、抗故障的第三方。具体而言,智能合约验证分析结果上签名的正确性。
\end{itemize}

\subsection{威胁模型}\label{subsec:threatmodel}
% 系统包含数据提供方、模型提供方、云服务提供方以及数据适用方四个角色,四个角色之间是相互不信任的。

系统包含四个不同的角色:数据提供方、模型提供方、云服务提供方和数据使用方,它们本质上相互不信任。
% 在数据分析的过程中,存在以下攻击:
% 数据窃取攻击:模型提供方的模型逻辑存在窃取原始数据或者中间数据的恶意代码;云服务提供方通过提供的硬件或软件资源窃取数据;数据使用方登录进服务器窃取数据。
% 数据伪造攻击:数据提供方提供的数据和声称的不一致。
% 数据滥用攻击:数据提供方的数据被使用在不同的模型、不同的云服务器上。
% 结果伪造攻击:数据分析的结果并不是模型如实运行得到的。
% 结果窃取攻击:数据分析的结果被数据提供方、模型提供方、云服务提供方或未知的攻击者获取。
在数据分析过程中,以下攻击普遍存在:
\begin{itemize}
    \item 数据窃取攻击:云服务提供方可能通过他们提供的硬件或软件资源窃取数据;数据使用方可能登录服务器窃取数据。模型提供方提供的程序可能包含旨在窃取原始或中间数据的恶意代码。
    \item 数据伪造攻击:数据提供方提供的数据与声称的不一致。
    \item 数据滥用攻击:数据提供方的数据在没有适当授权的情况下被用于不同的模型和云服务器。
    \item 结果伪造攻击:数据分析的结果并不准确反映模型的真实执行。
    \item 结果窃取攻击:数据分析的结果被数据提供方、模型提供方、云服务提供方或未知的攻击者非法获取。
\end{itemize}

我们在本文中不解决Intel SGX的侧信道攻击。我们假设Intel SGX的硬件功能如广告所示,确保enclave内的代码保持不变,内部变量的值受到保护,免受直接内存访问。值得注意的是,我们对Intel的依赖仅限于设置过程中的一次性交互;不需要与Intel服务器或DCAP进行进一步联系。相比之下,以前的方法依赖远程认证,要求Intel服务器或DCAP在每个数据分析任务中持续保持完整性和可用性。

我们假设存在抗碰撞哈希函数和安全签名及加密方案。具体而言,我们的签名包含\textit{nonce},确保有效的签名对$(msg, signed_{msg})$不能由没有私钥的对手生成,使历史签名无效。

\subsection{设计选择}\label{subsec:designchoice}
为了应对已识别的攻击,我们做出了以下设计选择。
% 抵御数据窃取攻击:所有存在于非可信环境(例如网络传输、云服务器的非TEE部分)的数据都经过加密处理,以防止数据使用方或云服务提供商直接窃取数据。模型提供方提供的分析程序,在进行数据分析前进行代码静态分析,防止模型提供方通过数据分析结果窃取数据。
% 抵御数据伪造攻击:执行数据分析前,检查数据哈希与声称的哈希是否一致。
% 抵御数据滥用攻击:引入数字签名技术,数据提供方用私钥签名使用数据的平台、模型以及模型的参数,在执行数据分析前,会验证该签名的有效性。
% 抵御结果伪造攻击:将数据使用方的私钥加密地传输到云服务器的可信执行环境中,并用这个私钥签名结果。当数据使用方拿到结果时,也会得到该私钥对结果的签名,若签名验证通过,说明结果确实是由可信执行环境中得到的,故结果未经伪造。这一思想类似于零知识证明。
% 抵御结果窃取攻击:数据分析结果由数据使用方的公钥加密,只能被数据使用方查看。
\begin{itemize}
    \item 抵御数据窃取攻击:所有存在于非可信环境(例如网络传输、云服务器的非TEE部分)的数据都经过加密处理,以防止数据使用方或云服务提供商直接窃取数据。模型提供方提供的分析程序,在进行数据分析前进行代码静态分析,防止模型提供方通过数据分析结果窃取数据。
    \item 抵御数据伪造攻击:执行数据分析前,检查数据哈希与声称的哈希是否一致。
    \item 抵御数据滥用攻击:引入数字签名技术,数据提供方用私钥签名使用数据的平台、模型以及模型的参数,在执行数据分析前,会验证该签名的有效性。
    \item 抵御结果伪造攻击:将数据使用方的私钥加密地传输到云服务器的可信执行环境中,并用这个私钥签名结果。当数据使用方拿到结果时,也会得到该私钥对结果的签名,若签名验证通过,说明结果确实是由可信执行环境中得到的,故结果未经伪造。这一思想类似于零知识证明。
    \item 抵御结果窃取攻击:数据分析结果由数据使用方的公钥加密,只能被数据使用方查看。
\end{itemize} 