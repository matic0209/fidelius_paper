\section{相关工作}
GXS作为一个基于区块链的数据交易平台运行,记录并促进买卖双方之间的交易,双方在相互同意后通过私有渠道交换数据。
AccountTrade使用区块链将分析器与数据提供方配对,并通过可执行的协议和数据索引确保生态系统安全,这些协议和索引检测并惩罚不诚实的行为。
Zhao等人开发了一个专注于提供方隐私的数据交易系统,采用环签名和双重认证来保护交易并防止未授权访问。

在数据共享系统中,Shen等人开发了一个确保数据完整性和机密性的方案。
Zuo等人引入了一个系统,使用代理重加密和密钥分离高效保护和撤销卖方的密钥,通过基于属性的加密增强数据保护。
为了缓解恶意代理参与数据泄露,Guo等人提出了可问责代理重加密(APRE),这是一个检测和解决重加密密钥滥用的框架,进一步验证了其在DBDH假设下的CPA安全性和可问责性。
Deng等人形式化了一个基于身份的加密转换(IBET)模型,用于与数据所有者初始指定之外的额外接收者共享加密数据。

SDTE引入了一个基于区块链的数据分析平台,数据提供方加密数据并上传到区块链。数据使用方选择这些数据,形成分析合约并请求服务。获得批准后,提供方通过Intel SGX远程认证将解密密钥发送给可信节点,然后该节点解密数据并在Intel SGX保护的以太坊虚拟机(EVM)中运行分析,然后将结果上传到结算合约。
然而,SDTE的加密数据上传对区块链施加了显著的存储需求,并且作为以太坊智能合约运行分析引入了性能挑战。在EVM上运行复杂的算法如\textit{k}-NN,由于数据规模和任务复杂性,通常是不实际的。

PrivacyGuard通过将链上分析转移到链外可信执行环境(TEE)来缓解以太坊虚拟机(EVM)的性能限制,允许数据提供方设置限制未授权数据使用的策略。然而,它仍然遇到EVM性能的挑战。类似地,Sterling引入了一个利用机器学习进行数据分析的去中心化数据市场,但也像SDTE和PrivacyGuard一样对区块链施加了显著的存储需求。此外,隐私担忧或法规可能阻止敏感数据的上传。 