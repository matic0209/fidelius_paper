\section{设计}\label{sec:design}

\subsection{概述}
本节简要概述Fidelius中使用的enclaves和符号,以及其架构和工作流程。

% 英文版原始内容(如无legend相关内容则无需更改)

\subsubsection{Enclaves和符号}
在本文中,我们定义以下enclaves和符号:
\begin{itemize}
    \item EKeyMgr。Enclave密钥管理器管理非对称密钥,处理创建、删除并提供基本密码学功能,包括消息加密、解密、签名和签名验证。
    \item EAnalyzer。Enclave分析器,一个Intel SGX Enclave格式的分析程序,确保程序在执行期间保持不变。
    \item $(PK_{DP},SK_{DP})$,数据提供方的非对称密钥对。
    \item $(PK_{DU},SK_{DU})$,数据使用方的非对称密钥对。
    \item $(PK_{CSP}, SK_{CSP})$表示云服务提供方的非对称密钥对,由EKeyMgr生成。私钥$SK_{CSP}$安全地存储在enclave内,确保云服务提供商无法提取。
    \item $H(\cdot)$表示哈希函数。
    \item $Enc(PK, msg)$表示使用$PK$加密$msg$。
    \item $Dec(SK, cipher)$表示使用$SK$解密$cipher$。
    \item $Sign(SK, msg)$表示使用$SK$对$msg$进行签名。
    \item $Verf(PK, msg, sig)$表示使用$PK$和$msg$验证$sig$。
    \item $\mathcal{F}(SK, PK_{CSP}, H_{EA})$表示使用$PK_{CSP}$将$SK$转发到EKeyMgr的过程。在此上下文中,$SK$在EAnalyzer中使用,由$H_{EA}$标识。函数$\mathcal{F}(\cdot)$涉及$Sign(SK, concat(PK_{CSP}, H_{EA}))$和$Enc(PK_{CSP}, SK)$。
\end{itemize}

\subsubsection{架构和工作流程}

Fidelius架构和数据分析工作流程如图~\ref{fig:arch}所示。
该过程从设置阶段开始,其中CSP中的EKeyMgr生成非对称密钥对。公钥$PK_{CSP}$通过Intel的远程认证服务进行验证,而私钥$SK_{CSP}$安全地存储在EKeyMgr内。同时,数据提供方(DP)和数据使用方(DU)各自生成自己的非对称密钥对,并保留这些密钥。此外,DP使用她的公钥$PK_{DP}$加密数据,并为DU的使用准备这些加密数据。

DP将加密数据上传到CSP,并使用$\mathcal{F}(SK_{DP}, PK_{CSP}, H_{EA})$将她的私钥$SK_{DP}$转发给EKeyMgr。类似地,DU将分析程序上传到CSP,并使用相同的方法将她的私钥$SK_{DU}$转发给EKeyMgr。重要的是要注意,明文私钥($SK_{DP}$和$SK_{DU}$)只能在EKeyMgr内解密。

一旦分析程序、加密数据和私钥准备就绪,数据分析任务就开始了。首先根据隐私描述语言(PDL)定义的规则检查分析程序。任何未通过此检查的程序都会导致分析任务立即终止。随后,EAnalyzer加载并解密加密数据。然后验证解密数据的哈希值;如果与声称的哈希不匹配,EAnalyzer立即停止以防止处理被篡改或欺诈的数据。如果数据得到验证,EAnalyzer继续进行主要分析,最终使用$PK_{DU}$产生加密结果。此外,使用$SK_{DU}$对结果进行签名。由于$SK_{DU}$位于EKeyMgr内,EAnalyzer通过本地认证建立安全通道来请求$SK_{DU}$。

% 英文版原始内容(如无legend相关内容则无需更改)

\subsection{隐私描述语言}
为了防止分析结果中的数据泄露,我们开发了一种隐私描述语言(PDL)来定义数据使用规则。使用静态二进制分析来确保分析程序遵循这些PDL定义的规则。如图~\ref{fig:pdl}所示,代码片段使用PDL来指定在Iris数据集上应用KMeans算法的规则。然后将此代码转换为LLVM的中间表示(IR),接着进行符号执行~\cite{king1976symbolic,baldoni2018survey}以得出PDL描述的状态$S$。同时,我们在分析中使用GTIRB~\cite{schulte2019gtirb}作为中间表示。将分析程序转换为此格式后,应用符号执行来获得每个输出变量的状态$S^{\prime}$,确保$S^{\prime}$是$S$的子集,从而确认合规性。

\subsection{可信和可验证的结果}
算法~\ref{algo:analysis_enclave}概述了EKeyMgr和EAnalyzer的主要操作。在EAnalyzer内生成的加密分析结果及其签名被传输到区块链。在那里,任何人都可以验证签名的有效性。有效的签名确认DP成功执行了分析程序并获得了正确的结果,因为签名源自EAnalyzer内部。这种机制防止攻击者在未执行分析程序的情况下伪造有效签名,从而确保数据分析过程的完整性。DU可以从区块链下载加密结果,并使用$SK_{DU}$解密以获得明文分析结果。
% 英文版原始算法内容(如无legend相关内容则无需更改)


\subsection{一次性远程认证}
我们实施一次性远程认证过程,如算法~\ref{algo:attestation}所述,从一开始就建立安全和可信的环境。在设置阶段,云服务提供商(CSP)通过与Intel认证服务进行远程认证来验证EKeyMgr生成的公钥$PK_{CSP}$的授权。这种彻底的验证确认了CSP凭据的完整性和合法性。一旦验证完成,CSP上的后续分析任务不再需要远程认证。DP和DU转发的私钥($SK$)通过本地认证从EKeyMgr安全传输到EAnalyzer。 